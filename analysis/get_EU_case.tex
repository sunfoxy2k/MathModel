\documentclass[11pt]{article}

    \usepackage[breakable]{tcolorbox}
    \usepackage{parskip} % Stop auto-indenting (to mimic markdown behaviour)
    
    \usepackage{iftex}
    \ifPDFTeX
    	\usepackage[T1]{fontenc}
    	\usepackage{mathpazo}
    \else
    	\usepackage{fontspec}
    \fi

    % Basic figure setup, for now with no caption control since it's done
    % automatically by Pandoc (which extracts ![](path) syntax from Markdown).
    \usepackage{graphicx}
    % Maintain compatibility with old templates. Remove in nbconvert 6.0
    \let\Oldincludegraphics\includegraphics
    % Ensure that by default, figures have no caption (until we provide a
    % proper Figure object with a Caption API and a way to capture that
    % in the conversion process - todo).
    \usepackage{caption}
    \DeclareCaptionFormat{nocaption}{}
    \captionsetup{format=nocaption,aboveskip=0pt,belowskip=0pt}

    \usepackage[Export]{adjustbox} % Used to constrain images to a maximum size
    \adjustboxset{max size={0.9\linewidth}{0.9\paperheight}}
    \usepackage{float}
    \floatplacement{figure}{H} % forces figures to be placed at the correct location
    \usepackage{xcolor} % Allow colors to be defined
    \usepackage{enumerate} % Needed for markdown enumerations to work
    \usepackage{geometry} % Used to adjust the document margins
    \usepackage{amsmath} % Equations
    \usepackage{amssymb} % Equations
    \usepackage{textcomp} % defines textquotesingle
    % Hack from http://tex.stackexchange.com/a/47451/13684:
    \AtBeginDocument{%
        \def\PYZsq{\textquotesingle}% Upright quotes in Pygmentized code
    }
    \usepackage{upquote} % Upright quotes for verbatim code
    \usepackage{eurosym} % defines \euro
    \usepackage[mathletters]{ucs} % Extended unicode (utf-8) support
    \usepackage{fancyvrb} % verbatim replacement that allows latex
    \usepackage{grffile} % extends the file name processing of package graphics 
                         % to support a larger range
    \makeatletter % fix for grffile with XeLaTeX
    \def\Gread@@xetex#1{%
      \IfFileExists{"\Gin@base".bb}%
      {\Gread@eps{\Gin@base.bb}}%
      {\Gread@@xetex@aux#1}%
    }
    \makeatother

    % The hyperref package gives us a pdf with properly built
    % internal navigation ('pdf bookmarks' for the table of contents,
    % internal cross-reference links, web links for URLs, etc.)
    \usepackage{hyperref}
    % The default LaTeX title has an obnoxious amount of whitespace. By default,
    % titling removes some of it. It also provides customization options.
    \usepackage{titling}
    \usepackage{longtable} % longtable support required by pandoc >1.10
    \usepackage{booktabs}  % table support for pandoc > 1.12.2
    \usepackage[inline]{enumitem} % IRkernel/repr support (it uses the enumerate* environment)
    \usepackage[normalem]{ulem} % ulem is needed to support strikethroughs (\sout)
                                % normalem makes italics be italics, not underlines
    \usepackage{mathrsfs}
    

    
    % Colors for the hyperref package
    \definecolor{urlcolor}{rgb}{0,.145,.698}
    \definecolor{linkcolor}{rgb}{.71,0.21,0.01}
    \definecolor{citecolor}{rgb}{.12,.54,.11}

    % ANSI colors
    \definecolor{ansi-black}{HTML}{3E424D}
    \definecolor{ansi-black-intense}{HTML}{282C36}
    \definecolor{ansi-red}{HTML}{E75C58}
    \definecolor{ansi-red-intense}{HTML}{B22B31}
    \definecolor{ansi-green}{HTML}{00A250}
    \definecolor{ansi-green-intense}{HTML}{007427}
    \definecolor{ansi-yellow}{HTML}{DDB62B}
    \definecolor{ansi-yellow-intense}{HTML}{B27D12}
    \definecolor{ansi-blue}{HTML}{208FFB}
    \definecolor{ansi-blue-intense}{HTML}{0065CA}
    \definecolor{ansi-magenta}{HTML}{D160C4}
    \definecolor{ansi-magenta-intense}{HTML}{A03196}
    \definecolor{ansi-cyan}{HTML}{60C6C8}
    \definecolor{ansi-cyan-intense}{HTML}{258F8F}
    \definecolor{ansi-white}{HTML}{C5C1B4}
    \definecolor{ansi-white-intense}{HTML}{A1A6B2}
    \definecolor{ansi-default-inverse-fg}{HTML}{FFFFFF}
    \definecolor{ansi-default-inverse-bg}{HTML}{000000}

    % commands and environments needed by pandoc snippets
    % extracted from the output of `pandoc -s`
    \providecommand{\tightlist}{%
      \setlength{\itemsep}{0pt}\setlength{\parskip}{0pt}}
    \DefineVerbatimEnvironment{Highlighting}{Verbatim}{commandchars=\\\{\}}
    % Add ',fontsize=\small' for more characters per line
    \newenvironment{Shaded}{}{}
    \newcommand{\KeywordTok}[1]{\textcolor[rgb]{0.00,0.44,0.13}{\textbf{{#1}}}}
    \newcommand{\DataTypeTok}[1]{\textcolor[rgb]{0.56,0.13,0.00}{{#1}}}
    \newcommand{\DecValTok}[1]{\textcolor[rgb]{0.25,0.63,0.44}{{#1}}}
    \newcommand{\BaseNTok}[1]{\textcolor[rgb]{0.25,0.63,0.44}{{#1}}}
    \newcommand{\FloatTok}[1]{\textcolor[rgb]{0.25,0.63,0.44}{{#1}}}
    \newcommand{\CharTok}[1]{\textcolor[rgb]{0.25,0.44,0.63}{{#1}}}
    \newcommand{\StringTok}[1]{\textcolor[rgb]{0.25,0.44,0.63}{{#1}}}
    \newcommand{\CommentTok}[1]{\textcolor[rgb]{0.38,0.63,0.69}{\textit{{#1}}}}
    \newcommand{\OtherTok}[1]{\textcolor[rgb]{0.00,0.44,0.13}{{#1}}}
    \newcommand{\AlertTok}[1]{\textcolor[rgb]{1.00,0.00,0.00}{\textbf{{#1}}}}
    \newcommand{\FunctionTok}[1]{\textcolor[rgb]{0.02,0.16,0.49}{{#1}}}
    \newcommand{\RegionMarkerTok}[1]{{#1}}
    \newcommand{\ErrorTok}[1]{\textcolor[rgb]{1.00,0.00,0.00}{\textbf{{#1}}}}
    \newcommand{\NormalTok}[1]{{#1}}
    
    % Additional commands for more recent versions of Pandoc
    \newcommand{\ConstantTok}[1]{\textcolor[rgb]{0.53,0.00,0.00}{{#1}}}
    \newcommand{\SpecialCharTok}[1]{\textcolor[rgb]{0.25,0.44,0.63}{{#1}}}
    \newcommand{\VerbatimStringTok}[1]{\textcolor[rgb]{0.25,0.44,0.63}{{#1}}}
    \newcommand{\SpecialStringTok}[1]{\textcolor[rgb]{0.73,0.40,0.53}{{#1}}}
    \newcommand{\ImportTok}[1]{{#1}}
    \newcommand{\DocumentationTok}[1]{\textcolor[rgb]{0.73,0.13,0.13}{\textit{{#1}}}}
    \newcommand{\AnnotationTok}[1]{\textcolor[rgb]{0.38,0.63,0.69}{\textbf{\textit{{#1}}}}}
    \newcommand{\CommentVarTok}[1]{\textcolor[rgb]{0.38,0.63,0.69}{\textbf{\textit{{#1}}}}}
    \newcommand{\VariableTok}[1]{\textcolor[rgb]{0.10,0.09,0.49}{{#1}}}
    \newcommand{\ControlFlowTok}[1]{\textcolor[rgb]{0.00,0.44,0.13}{\textbf{{#1}}}}
    \newcommand{\OperatorTok}[1]{\textcolor[rgb]{0.40,0.40,0.40}{{#1}}}
    \newcommand{\BuiltInTok}[1]{{#1}}
    \newcommand{\ExtensionTok}[1]{{#1}}
    \newcommand{\PreprocessorTok}[1]{\textcolor[rgb]{0.74,0.48,0.00}{{#1}}}
    \newcommand{\AttributeTok}[1]{\textcolor[rgb]{0.49,0.56,0.16}{{#1}}}
    \newcommand{\InformationTok}[1]{\textcolor[rgb]{0.38,0.63,0.69}{\textbf{\textit{{#1}}}}}
    \newcommand{\WarningTok}[1]{\textcolor[rgb]{0.38,0.63,0.69}{\textbf{\textit{{#1}}}}}
    
    
    % Define a nice break command that doesn't care if a line doesn't already
    % exist.
    \def\br{\hspace*{\fill} \\* }
    % Math Jax compatibility definitions
    \def\gt{>}
    \def\lt{<}
    \let\Oldtex\TeX
    \let\Oldlatex\LaTeX
    \renewcommand{\TeX}{\textrm{\Oldtex}}
    \renewcommand{\LaTeX}{\textrm{\Oldlatex}}
    % Document parameters
    % Document title
    \title{get\_EU\_case}
    
    
    
    
    
% Pygments definitions
\makeatletter
\def\PY@reset{\let\PY@it=\relax \let\PY@bf=\relax%
    \let\PY@ul=\relax \let\PY@tc=\relax%
    \let\PY@bc=\relax \let\PY@ff=\relax}
\def\PY@tok#1{\csname PY@tok@#1\endcsname}
\def\PY@toks#1+{\ifx\relax#1\empty\else%
    \PY@tok{#1}\expandafter\PY@toks\fi}
\def\PY@do#1{\PY@bc{\PY@tc{\PY@ul{%
    \PY@it{\PY@bf{\PY@ff{#1}}}}}}}
\def\PY#1#2{\PY@reset\PY@toks#1+\relax+\PY@do{#2}}

\expandafter\def\csname PY@tok@w\endcsname{\def\PY@tc##1{\textcolor[rgb]{0.73,0.73,0.73}{##1}}}
\expandafter\def\csname PY@tok@c\endcsname{\let\PY@it=\textit\def\PY@tc##1{\textcolor[rgb]{0.25,0.50,0.50}{##1}}}
\expandafter\def\csname PY@tok@cp\endcsname{\def\PY@tc##1{\textcolor[rgb]{0.74,0.48,0.00}{##1}}}
\expandafter\def\csname PY@tok@k\endcsname{\let\PY@bf=\textbf\def\PY@tc##1{\textcolor[rgb]{0.00,0.50,0.00}{##1}}}
\expandafter\def\csname PY@tok@kp\endcsname{\def\PY@tc##1{\textcolor[rgb]{0.00,0.50,0.00}{##1}}}
\expandafter\def\csname PY@tok@kt\endcsname{\def\PY@tc##1{\textcolor[rgb]{0.69,0.00,0.25}{##1}}}
\expandafter\def\csname PY@tok@o\endcsname{\def\PY@tc##1{\textcolor[rgb]{0.40,0.40,0.40}{##1}}}
\expandafter\def\csname PY@tok@ow\endcsname{\let\PY@bf=\textbf\def\PY@tc##1{\textcolor[rgb]{0.67,0.13,1.00}{##1}}}
\expandafter\def\csname PY@tok@nb\endcsname{\def\PY@tc##1{\textcolor[rgb]{0.00,0.50,0.00}{##1}}}
\expandafter\def\csname PY@tok@nf\endcsname{\def\PY@tc##1{\textcolor[rgb]{0.00,0.00,1.00}{##1}}}
\expandafter\def\csname PY@tok@nc\endcsname{\let\PY@bf=\textbf\def\PY@tc##1{\textcolor[rgb]{0.00,0.00,1.00}{##1}}}
\expandafter\def\csname PY@tok@nn\endcsname{\let\PY@bf=\textbf\def\PY@tc##1{\textcolor[rgb]{0.00,0.00,1.00}{##1}}}
\expandafter\def\csname PY@tok@ne\endcsname{\let\PY@bf=\textbf\def\PY@tc##1{\textcolor[rgb]{0.82,0.25,0.23}{##1}}}
\expandafter\def\csname PY@tok@nv\endcsname{\def\PY@tc##1{\textcolor[rgb]{0.10,0.09,0.49}{##1}}}
\expandafter\def\csname PY@tok@no\endcsname{\def\PY@tc##1{\textcolor[rgb]{0.53,0.00,0.00}{##1}}}
\expandafter\def\csname PY@tok@nl\endcsname{\def\PY@tc##1{\textcolor[rgb]{0.63,0.63,0.00}{##1}}}
\expandafter\def\csname PY@tok@ni\endcsname{\let\PY@bf=\textbf\def\PY@tc##1{\textcolor[rgb]{0.60,0.60,0.60}{##1}}}
\expandafter\def\csname PY@tok@na\endcsname{\def\PY@tc##1{\textcolor[rgb]{0.49,0.56,0.16}{##1}}}
\expandafter\def\csname PY@tok@nt\endcsname{\let\PY@bf=\textbf\def\PY@tc##1{\textcolor[rgb]{0.00,0.50,0.00}{##1}}}
\expandafter\def\csname PY@tok@nd\endcsname{\def\PY@tc##1{\textcolor[rgb]{0.67,0.13,1.00}{##1}}}
\expandafter\def\csname PY@tok@s\endcsname{\def\PY@tc##1{\textcolor[rgb]{0.73,0.13,0.13}{##1}}}
\expandafter\def\csname PY@tok@sd\endcsname{\let\PY@it=\textit\def\PY@tc##1{\textcolor[rgb]{0.73,0.13,0.13}{##1}}}
\expandafter\def\csname PY@tok@si\endcsname{\let\PY@bf=\textbf\def\PY@tc##1{\textcolor[rgb]{0.73,0.40,0.53}{##1}}}
\expandafter\def\csname PY@tok@se\endcsname{\let\PY@bf=\textbf\def\PY@tc##1{\textcolor[rgb]{0.73,0.40,0.13}{##1}}}
\expandafter\def\csname PY@tok@sr\endcsname{\def\PY@tc##1{\textcolor[rgb]{0.73,0.40,0.53}{##1}}}
\expandafter\def\csname PY@tok@ss\endcsname{\def\PY@tc##1{\textcolor[rgb]{0.10,0.09,0.49}{##1}}}
\expandafter\def\csname PY@tok@sx\endcsname{\def\PY@tc##1{\textcolor[rgb]{0.00,0.50,0.00}{##1}}}
\expandafter\def\csname PY@tok@m\endcsname{\def\PY@tc##1{\textcolor[rgb]{0.40,0.40,0.40}{##1}}}
\expandafter\def\csname PY@tok@gh\endcsname{\let\PY@bf=\textbf\def\PY@tc##1{\textcolor[rgb]{0.00,0.00,0.50}{##1}}}
\expandafter\def\csname PY@tok@gu\endcsname{\let\PY@bf=\textbf\def\PY@tc##1{\textcolor[rgb]{0.50,0.00,0.50}{##1}}}
\expandafter\def\csname PY@tok@gd\endcsname{\def\PY@tc##1{\textcolor[rgb]{0.63,0.00,0.00}{##1}}}
\expandafter\def\csname PY@tok@gi\endcsname{\def\PY@tc##1{\textcolor[rgb]{0.00,0.63,0.00}{##1}}}
\expandafter\def\csname PY@tok@gr\endcsname{\def\PY@tc##1{\textcolor[rgb]{1.00,0.00,0.00}{##1}}}
\expandafter\def\csname PY@tok@ge\endcsname{\let\PY@it=\textit}
\expandafter\def\csname PY@tok@gs\endcsname{\let\PY@bf=\textbf}
\expandafter\def\csname PY@tok@gp\endcsname{\let\PY@bf=\textbf\def\PY@tc##1{\textcolor[rgb]{0.00,0.00,0.50}{##1}}}
\expandafter\def\csname PY@tok@go\endcsname{\def\PY@tc##1{\textcolor[rgb]{0.53,0.53,0.53}{##1}}}
\expandafter\def\csname PY@tok@gt\endcsname{\def\PY@tc##1{\textcolor[rgb]{0.00,0.27,0.87}{##1}}}
\expandafter\def\csname PY@tok@err\endcsname{\def\PY@bc##1{\setlength{\fboxsep}{0pt}\fcolorbox[rgb]{1.00,0.00,0.00}{1,1,1}{\strut ##1}}}
\expandafter\def\csname PY@tok@kc\endcsname{\let\PY@bf=\textbf\def\PY@tc##1{\textcolor[rgb]{0.00,0.50,0.00}{##1}}}
\expandafter\def\csname PY@tok@kd\endcsname{\let\PY@bf=\textbf\def\PY@tc##1{\textcolor[rgb]{0.00,0.50,0.00}{##1}}}
\expandafter\def\csname PY@tok@kn\endcsname{\let\PY@bf=\textbf\def\PY@tc##1{\textcolor[rgb]{0.00,0.50,0.00}{##1}}}
\expandafter\def\csname PY@tok@kr\endcsname{\let\PY@bf=\textbf\def\PY@tc##1{\textcolor[rgb]{0.00,0.50,0.00}{##1}}}
\expandafter\def\csname PY@tok@bp\endcsname{\def\PY@tc##1{\textcolor[rgb]{0.00,0.50,0.00}{##1}}}
\expandafter\def\csname PY@tok@fm\endcsname{\def\PY@tc##1{\textcolor[rgb]{0.00,0.00,1.00}{##1}}}
\expandafter\def\csname PY@tok@vc\endcsname{\def\PY@tc##1{\textcolor[rgb]{0.10,0.09,0.49}{##1}}}
\expandafter\def\csname PY@tok@vg\endcsname{\def\PY@tc##1{\textcolor[rgb]{0.10,0.09,0.49}{##1}}}
\expandafter\def\csname PY@tok@vi\endcsname{\def\PY@tc##1{\textcolor[rgb]{0.10,0.09,0.49}{##1}}}
\expandafter\def\csname PY@tok@vm\endcsname{\def\PY@tc##1{\textcolor[rgb]{0.10,0.09,0.49}{##1}}}
\expandafter\def\csname PY@tok@sa\endcsname{\def\PY@tc##1{\textcolor[rgb]{0.73,0.13,0.13}{##1}}}
\expandafter\def\csname PY@tok@sb\endcsname{\def\PY@tc##1{\textcolor[rgb]{0.73,0.13,0.13}{##1}}}
\expandafter\def\csname PY@tok@sc\endcsname{\def\PY@tc##1{\textcolor[rgb]{0.73,0.13,0.13}{##1}}}
\expandafter\def\csname PY@tok@dl\endcsname{\def\PY@tc##1{\textcolor[rgb]{0.73,0.13,0.13}{##1}}}
\expandafter\def\csname PY@tok@s2\endcsname{\def\PY@tc##1{\textcolor[rgb]{0.73,0.13,0.13}{##1}}}
\expandafter\def\csname PY@tok@sh\endcsname{\def\PY@tc##1{\textcolor[rgb]{0.73,0.13,0.13}{##1}}}
\expandafter\def\csname PY@tok@s1\endcsname{\def\PY@tc##1{\textcolor[rgb]{0.73,0.13,0.13}{##1}}}
\expandafter\def\csname PY@tok@mb\endcsname{\def\PY@tc##1{\textcolor[rgb]{0.40,0.40,0.40}{##1}}}
\expandafter\def\csname PY@tok@mf\endcsname{\def\PY@tc##1{\textcolor[rgb]{0.40,0.40,0.40}{##1}}}
\expandafter\def\csname PY@tok@mh\endcsname{\def\PY@tc##1{\textcolor[rgb]{0.40,0.40,0.40}{##1}}}
\expandafter\def\csname PY@tok@mi\endcsname{\def\PY@tc##1{\textcolor[rgb]{0.40,0.40,0.40}{##1}}}
\expandafter\def\csname PY@tok@il\endcsname{\def\PY@tc##1{\textcolor[rgb]{0.40,0.40,0.40}{##1}}}
\expandafter\def\csname PY@tok@mo\endcsname{\def\PY@tc##1{\textcolor[rgb]{0.40,0.40,0.40}{##1}}}
\expandafter\def\csname PY@tok@ch\endcsname{\let\PY@it=\textit\def\PY@tc##1{\textcolor[rgb]{0.25,0.50,0.50}{##1}}}
\expandafter\def\csname PY@tok@cm\endcsname{\let\PY@it=\textit\def\PY@tc##1{\textcolor[rgb]{0.25,0.50,0.50}{##1}}}
\expandafter\def\csname PY@tok@cpf\endcsname{\let\PY@it=\textit\def\PY@tc##1{\textcolor[rgb]{0.25,0.50,0.50}{##1}}}
\expandafter\def\csname PY@tok@c1\endcsname{\let\PY@it=\textit\def\PY@tc##1{\textcolor[rgb]{0.25,0.50,0.50}{##1}}}
\expandafter\def\csname PY@tok@cs\endcsname{\let\PY@it=\textit\def\PY@tc##1{\textcolor[rgb]{0.25,0.50,0.50}{##1}}}

\def\PYZbs{\char`\\}
\def\PYZus{\char`\_}
\def\PYZob{\char`\{}
\def\PYZcb{\char`\}}
\def\PYZca{\char`\^}
\def\PYZam{\char`\&}
\def\PYZlt{\char`\<}
\def\PYZgt{\char`\>}
\def\PYZsh{\char`\#}
\def\PYZpc{\char`\%}
\def\PYZdl{\char`\$}
\def\PYZhy{\char`\-}
\def\PYZsq{\char`\'}
\def\PYZdq{\char`\"}
\def\PYZti{\char`\~}
% for compatibility with earlier versions
\def\PYZat{@}
\def\PYZlb{[}
\def\PYZrb{]}
\makeatother


    % For linebreaks inside Verbatim environment from package fancyvrb. 
    \makeatletter
        \newbox\Wrappedcontinuationbox 
        \newbox\Wrappedvisiblespacebox 
        \newcommand*\Wrappedvisiblespace {\textcolor{red}{\textvisiblespace}} 
        \newcommand*\Wrappedcontinuationsymbol {\textcolor{red}{\llap{\tiny$\m@th\hookrightarrow$}}} 
        \newcommand*\Wrappedcontinuationindent {3ex } 
        \newcommand*\Wrappedafterbreak {\kern\Wrappedcontinuationindent\copy\Wrappedcontinuationbox} 
        % Take advantage of the already applied Pygments mark-up to insert 
        % potential linebreaks for TeX processing. 
        %        {, <, #, %, $, ' and ": go to next line. 
        %        _, }, ^, &, >, - and ~: stay at end of broken line. 
        % Use of \textquotesingle for straight quote. 
        \newcommand*\Wrappedbreaksatspecials {% 
            \def\PYGZus{\discretionary{\char`\_}{\Wrappedafterbreak}{\char`\_}}% 
            \def\PYGZob{\discretionary{}{\Wrappedafterbreak\char`\{}{\char`\{}}% 
            \def\PYGZcb{\discretionary{\char`\}}{\Wrappedafterbreak}{\char`\}}}% 
            \def\PYGZca{\discretionary{\char`\^}{\Wrappedafterbreak}{\char`\^}}% 
            \def\PYGZam{\discretionary{\char`\&}{\Wrappedafterbreak}{\char`\&}}% 
            \def\PYGZlt{\discretionary{}{\Wrappedafterbreak\char`\<}{\char`\<}}% 
            \def\PYGZgt{\discretionary{\char`\>}{\Wrappedafterbreak}{\char`\>}}% 
            \def\PYGZsh{\discretionary{}{\Wrappedafterbreak\char`\#}{\char`\#}}% 
            \def\PYGZpc{\discretionary{}{\Wrappedafterbreak\char`\%}{\char`\%}}% 
            \def\PYGZdl{\discretionary{}{\Wrappedafterbreak\char`\$}{\char`\$}}% 
            \def\PYGZhy{\discretionary{\char`\-}{\Wrappedafterbreak}{\char`\-}}% 
            \def\PYGZsq{\discretionary{}{\Wrappedafterbreak\textquotesingle}{\textquotesingle}}% 
            \def\PYGZdq{\discretionary{}{\Wrappedafterbreak\char`\"}{\char`\"}}% 
            \def\PYGZti{\discretionary{\char`\~}{\Wrappedafterbreak}{\char`\~}}% 
        } 
        % Some characters . , ; ? ! / are not pygmentized. 
        % This macro makes them "active" and they will insert potential linebreaks 
        \newcommand*\Wrappedbreaksatpunct {% 
            \lccode`\~`\.\lowercase{\def~}{\discretionary{\hbox{\char`\.}}{\Wrappedafterbreak}{\hbox{\char`\.}}}% 
            \lccode`\~`\,\lowercase{\def~}{\discretionary{\hbox{\char`\,}}{\Wrappedafterbreak}{\hbox{\char`\,}}}% 
            \lccode`\~`\;\lowercase{\def~}{\discretionary{\hbox{\char`\;}}{\Wrappedafterbreak}{\hbox{\char`\;}}}% 
            \lccode`\~`\:\lowercase{\def~}{\discretionary{\hbox{\char`\:}}{\Wrappedafterbreak}{\hbox{\char`\:}}}% 
            \lccode`\~`\?\lowercase{\def~}{\discretionary{\hbox{\char`\?}}{\Wrappedafterbreak}{\hbox{\char`\?}}}% 
            \lccode`\~`\!\lowercase{\def~}{\discretionary{\hbox{\char`\!}}{\Wrappedafterbreak}{\hbox{\char`\!}}}% 
            \lccode`\~`\/\lowercase{\def~}{\discretionary{\hbox{\char`\/}}{\Wrappedafterbreak}{\hbox{\char`\/}}}% 
            \catcode`\.\active
            \catcode`\,\active 
            \catcode`\;\active
            \catcode`\:\active
            \catcode`\?\active
            \catcode`\!\active
            \catcode`\/\active 
            \lccode`\~`\~ 	
        }
    \makeatother

    \let\OriginalVerbatim=\Verbatim
    \makeatletter
    \renewcommand{\Verbatim}[1][1]{%
        %\parskip\z@skip
        \sbox\Wrappedcontinuationbox {\Wrappedcontinuationsymbol}%
        \sbox\Wrappedvisiblespacebox {\FV@SetupFont\Wrappedvisiblespace}%
        \def\FancyVerbFormatLine ##1{\hsize\linewidth
            \vtop{\raggedright\hyphenpenalty\z@\exhyphenpenalty\z@
                \doublehyphendemerits\z@\finalhyphendemerits\z@
                \strut ##1\strut}%
        }%
        % If the linebreak is at a space, the latter will be displayed as visible
        % space at end of first line, and a continuation symbol starts next line.
        % Stretch/shrink are however usually zero for typewriter font.
        \def\FV@Space {%
            \nobreak\hskip\z@ plus\fontdimen3\font minus\fontdimen4\font
            \discretionary{\copy\Wrappedvisiblespacebox}{\Wrappedafterbreak}
            {\kern\fontdimen2\font}%
        }%
        
        % Allow breaks at special characters using \PYG... macros.
        \Wrappedbreaksatspecials
        % Breaks at punctuation characters . , ; ? ! and / need catcode=\active 	
        \OriginalVerbatim[#1,codes*=\Wrappedbreaksatpunct]%
    }
    \makeatother

    % Exact colors from NB
    \definecolor{incolor}{HTML}{303F9F}
    \definecolor{outcolor}{HTML}{D84315}
    \definecolor{cellborder}{HTML}{CFCFCF}
    \definecolor{cellbackground}{HTML}{F7F7F7}
    
    % prompt
    \makeatletter
    \newcommand{\boxspacing}{\kern\kvtcb@left@rule\kern\kvtcb@boxsep}
    \makeatother
    \newcommand{\prompt}[4]{
        \ttfamily\llap{{\color{#2}[#3]:\hspace{3pt}#4}}\vspace{-\baselineskip}
    }
    

    
    % Prevent overflowing lines due to hard-to-break entities
    \sloppy 
    % Setup hyperref package
    \hypersetup{
      breaklinks=true,  % so long urls are correctly broken across lines
      colorlinks=true,
      urlcolor=urlcolor,
      linkcolor=linkcolor,
      citecolor=citecolor,
      }
    % Slightly bigger margins than the latex defaults
    
    \geometry{verbose,tmargin=1in,bmargin=1in,lmargin=1in,rmargin=1in}
    
    

\begin{document}
    
    \maketitle
    
    

    
    \hypertarget{description}{%
\section{Description}\label{description}}

    Get Covid case information in EU

    \hypertarget{environment}{%
\section{Environment}\label{environment}}

    \hypertarget{library-imports}{%
\subsection{Library Imports}\label{library-imports}}

    \begin{tcolorbox}[breakable, size=fbox, boxrule=1pt, pad at break*=1mm,colback=cellbackground, colframe=cellborder]
\prompt{In}{incolor}{1}{\boxspacing}
\begin{Verbatim}[commandchars=\\\{\}]
\PY{k+kn}{import} \PY{n+nn}{os}
\PY{k+kn}{import} \PY{n+nn}{gc}
\PY{n}{gc}\PY{o}{.}\PY{n}{enable}\PY{p}{(}\PY{p}{)}

\PY{k+kn}{import} \PY{n+nn}{numpy} \PY{k}{as} \PY{n+nn}{np}
\PY{k+kn}{import} \PY{n+nn}{pandas} \PY{k}{as} \PY{n+nn}{pd}

\PY{k+kn}{import} \PY{n+nn}{matplotlib}\PY{n+nn}{.}\PY{n+nn}{pyplot} \PY{k}{as} \PY{n+nn}{plt}
\PY{k+kn}{import} \PY{n+nn}{seaborn} \PY{k}{as} \PY{n+nn}{sns}
\PY{o}{\PYZpc{}}\PY{k}{matplotlib} inline
\PY{n}{plt}\PY{o}{.}\PY{n}{style}\PY{o}{.}\PY{n}{use}\PY{p}{(}\PY{l+s+s1}{\PYZsq{}}\PY{l+s+s1}{ggplot}\PY{l+s+s1}{\PYZsq{}}\PY{p}{)}
\end{Verbatim}
\end{tcolorbox}

    \hypertarget{local-imports}{%
\subsection{Local Imports}\label{local-imports}}

    \begin{tcolorbox}[breakable, size=fbox, boxrule=1pt, pad at break*=1mm,colback=cellbackground, colframe=cellborder]
\prompt{In}{incolor}{2}{\boxspacing}
\begin{Verbatim}[commandchars=\\\{\}]
\PY{k+kn}{from} \PY{n+nn}{phuc} \PY{k+kn}{import} \PY{n}{data\PYZus{}process}\PY{p}{,} \PY{n}{file}
\PY{k+kn}{from} \PY{n+nn}{phuc} \PY{k+kn}{import} \PY{n}{visualization} \PY{k}{as} \PY{n}{vs}
\end{Verbatim}
\end{tcolorbox}

    \hypertarget{reload-data-path}{%
\subsection{Reload data path}\label{reload-data-path}}

    \begin{tcolorbox}[breakable, size=fbox, boxrule=1pt, pad at break*=1mm,colback=cellbackground, colframe=cellborder]
\prompt{In}{incolor}{3}{\boxspacing}
\begin{Verbatim}[commandchars=\\\{\}]
\PY{k+kn}{from} \PY{n+nn}{phuc}\PY{n+nn}{.}\PY{n+nn}{file} \PY{k+kn}{import} \PY{n}{standard\PYZus{}template}

\PY{n}{standard\PYZus{}template}\PY{o}{.}\PY{n}{save\PYZus{}data\PYZus{}path}\PY{p}{(}\PY{n}{os}\PY{o}{.}\PY{n}{getcwd}\PY{p}{(}\PY{p}{)}\PY{o}{.}\PY{n}{split}\PY{p}{(}\PY{l+s+s1}{\PYZsq{}}\PY{l+s+s1}{/src}\PY{l+s+s1}{\PYZsq{}}\PY{p}{)}\PY{p}{[}\PY{l+m+mi}{0}\PY{p}{]}\PY{p}{)}
\end{Verbatim}
\end{tcolorbox}

    \begin{Verbatim}[commandchars=\\\{\}]
Loaded file data\_path.pkl
Directory  /home/phuc/Desktop/Work/Data Sience/Epidemic Modeling/MathModel
already exists
Saved file data\_path.pkl
    \end{Verbatim}

    \hypertarget{setup-pandas}{%
\subsection{Setup Pandas}\label{setup-pandas}}

    \begin{tcolorbox}[breakable, size=fbox, boxrule=1pt, pad at break*=1mm,colback=cellbackground, colframe=cellborder]
\prompt{In}{incolor}{4}{\boxspacing}
\begin{Verbatim}[commandchars=\\\{\}]
\PY{n}{pd}\PY{o}{.}\PY{n}{set\PYZus{}option}\PY{p}{(}\PY{l+s+s1}{\PYZsq{}}\PY{l+s+s1}{display.max\PYZus{}rows}\PY{l+s+s1}{\PYZsq{}}\PY{p}{,} \PY{l+m+mi}{200}\PY{p}{)}
\PY{n}{pd}\PY{o}{.}\PY{n}{set\PYZus{}option}\PY{p}{(}\PY{l+s+s1}{\PYZsq{}}\PY{l+s+s1}{display.max\PYZus{}columns}\PY{l+s+s1}{\PYZsq{}}\PY{p}{,} \PY{l+m+mi}{100}\PY{p}{)}
\end{Verbatim}
\end{tcolorbox}

    \hypertarget{set-up-data-path}{%
\subsection{Set up data path}\label{set-up-data-path}}

    \begin{tcolorbox}[breakable, size=fbox, boxrule=1pt, pad at break*=1mm,colback=cellbackground, colframe=cellborder]
\prompt{In}{incolor}{5}{\boxspacing}
\begin{Verbatim}[commandchars=\\\{\}]
\PY{n}{CURR\PYZus{}DIR} \PY{o}{=} \PY{n}{os}\PY{o}{.}\PY{n}{getcwd}\PY{p}{(}\PY{p}{)}\PY{o}{.}\PY{n}{split}\PY{p}{(}\PY{l+s+s1}{\PYZsq{}}\PY{l+s+s1}{/src}\PY{l+s+s1}{\PYZsq{}}\PY{p}{)}\PY{p}{[}\PY{l+m+mi}{0}\PY{p}{]}
\end{Verbatim}
\end{tcolorbox}

    \begin{tcolorbox}[breakable, size=fbox, boxrule=1pt, pad at break*=1mm,colback=cellbackground, colframe=cellborder]
\prompt{In}{incolor}{6}{\boxspacing}
\begin{Verbatim}[commandchars=\\\{\}]
\PY{n}{file\PYZus{}path} \PY{o}{=} \PY{n}{file}\PY{o}{.}\PY{n}{load\PYZus{}pickle}\PY{p}{(}\PY{n}{CURR\PYZus{}DIR} \PY{o}{+} \PY{l+s+s1}{\PYZsq{}}\PY{l+s+s1}{/data\PYZus{}path.pkl}\PY{l+s+s1}{\PYZsq{}}\PY{p}{)}\PY{p}{[}\PY{l+s+s1}{\PYZsq{}}\PY{l+s+s1}{FILES}\PY{l+s+s1}{\PYZsq{}}\PY{p}{]}
\PY{n}{dir\PYZus{}path} \PY{o}{=} \PY{n}{file}\PY{o}{.}\PY{n}{load\PYZus{}pickle}\PY{p}{(}\PY{n}{CURR\PYZus{}DIR} \PY{o}{+} \PY{l+s+s1}{\PYZsq{}}\PY{l+s+s1}{/data\PYZus{}path.pkl}\PY{l+s+s1}{\PYZsq{}}\PY{p}{)}\PY{p}{[}\PY{l+s+s1}{\PYZsq{}}\PY{l+s+s1}{DIRS}\PY{l+s+s1}{\PYZsq{}}\PY{p}{]}
\end{Verbatim}
\end{tcolorbox}

    \begin{Verbatim}[commandchars=\\\{\}]
Loaded file data\_path.pkl
Loaded file data\_path.pkl
    \end{Verbatim}

    \begin{tcolorbox}[breakable, size=fbox, boxrule=1pt, pad at break*=1mm,colback=cellbackground, colframe=cellborder]
\prompt{In}{incolor}{7}{\boxspacing}
\begin{Verbatim}[commandchars=\\\{\}]
\PY{c+c1}{\PYZsh{}for key in file\PYZus{}path: print(key)}
\PY{c+c1}{\PYZsh{}for key in dir\PYZus{}path: print(key)}
\end{Verbatim}
\end{tcolorbox}

    \hypertarget{code}{%
\section{Code}\label{code}}

    \hypertarget{john-hopskin-corona-virus-repository}{%
\section{JOHN HOPSKIN Corona-Virus
Repository}\label{john-hopskin-corona-virus-repository}}

    We use data from
COVID-19/csse\_covid\_19\_data/csse\_covid\_19\_time\_series/ that was
cloned from https://github.com/CSSEGISandData/COVID-19 in 08/07/2020

This folder contains daily time series summary tables, including
confirmed, deaths and recovered. All data is read in from the daily case
report. The time series tables are subject to be updated if inaccuracies
are identified in our historical data. The daily reports will not be
adjusted in these instances to maintain a record of raw data.

Two time series tables are for the US confirmed cases and deaths,
reported at the county level. They are named
time\_series\_covid19\_confirmed\_US.csv,
time\_series\_covid19\_deaths\_US.csv, respectively.

Three time series tables are for the global confirmed cases, recovered
cases and deaths. Australia, Canada and China are reported at the
province/state level. Dependencies of the Netherlands, the UK, France
and Denmark are listed under the province/state level. The US and other
countries are at the country level. The tables are renamed
time\_series\_covid19\_confirmed\_global.csv and
time\_series\_covid19\_deaths\_global.csv, and
time\_series\_covid19\_recovered\_global.csv, respectively.

    \hypertarget{general-information}{%
\subsection{General Information}\label{general-information}}

    We decide to use 3 data file:

\begin{itemize}
\tightlist
\item
  TIME\_SERIES\_COVID19\_CONFIRMED\_GLOBAL\_CSV
\item
  TIME\_SERIES\_COVID19\_RECOVERED\_GLOBAL\_CSV\_PATH
\item
  TIME\_SERIES\_COVID19\_DEATHS\_GLOBAL\_CSV\_PATH
\end{itemize}

Those files share some simlarites:

The Data about Corona-Virus from the JOHN HOPSKIN is contain the data
from more than 173 Countries/Regions recorded from 1/22/20 to 7/8/20
(MM/DD/YY) in the time, we cloned to our repos.

Only have Null value at the Province/State, which is reasonable.

    \begin{tcolorbox}[breakable, size=fbox, boxrule=1pt, pad at break*=1mm,colback=cellbackground, colframe=cellborder]
\prompt{In}{incolor}{13}{\boxspacing}
\begin{Verbatim}[commandchars=\\\{\}]
\PY{n}{dat\PYZus{}confirmed} \PY{o}{=} \PY{n}{data\PYZus{}process}\PY{o}{.}\PY{n}{read\PYZus{}csv}\PY{p}{(}\PY{n}{file\PYZus{}path}\PY{p}{[}\PYZbs{}
                    \PY{l+s+s1}{\PYZsq{}}\PY{l+s+s1}{TIME\PYZus{}SERIES\PYZus{}COVID19\PYZus{}CONFIRMED\PYZus{}GLOBAL\PYZus{}CSV\PYZus{}PATH}\PY{l+s+s1}{\PYZsq{}}\PY{p}{]}\PY{p}{)}
\end{Verbatim}
\end{tcolorbox}

    \begin{Verbatim}[commandchars=\\\{\}]
time\_series\_covid19\_confirmed\_global.csv 's information:
<class 'pandas.core.frame.DataFrame'>
RangeIndex: 266 entries, 0 to 265
Columns: 173 entries, Province/State to 7/8/20
dtypes: float64(2), int64(169), object(2)
memory usage: 359.6+ KB
None

    \end{Verbatim}

    \begin{tcolorbox}[breakable, size=fbox, boxrule=1pt, pad at break*=1mm,colback=cellbackground, colframe=cellborder]
\prompt{In}{incolor}{71}{\boxspacing}
\begin{Verbatim}[commandchars=\\\{\}]
\PY{n}{dat\PYZus{}recovered} \PY{o}{=} \PY{n}{data\PYZus{}process}\PY{o}{.}\PY{n}{read\PYZus{}csv}\PY{p}{(}\PY{n}{file\PYZus{}path}\PY{p}{[}\PYZbs{}
                    \PY{l+s+s1}{\PYZsq{}}\PY{l+s+s1}{TIME\PYZus{}SERIES\PYZus{}COVID19\PYZus{}RECOVERED\PYZus{}GLOBAL\PYZus{}CSV\PYZus{}PATH}\PY{l+s+s1}{\PYZsq{}}\PY{p}{]}\PY{p}{)}
\end{Verbatim}
\end{tcolorbox}

    \begin{Verbatim}[commandchars=\\\{\}]
time\_series\_covid19\_recovered\_global.csv 's information:
<class 'pandas.core.frame.DataFrame'>
RangeIndex: 253 entries, 0 to 252
Columns: 173 entries, Province/State to 7/8/20
dtypes: float64(2), int64(169), object(2)
memory usage: 342.1+ KB
None

    \end{Verbatim}

    \begin{tcolorbox}[breakable, size=fbox, boxrule=1pt, pad at break*=1mm,colback=cellbackground, colframe=cellborder]
\prompt{In}{incolor}{72}{\boxspacing}
\begin{Verbatim}[commandchars=\\\{\}]
\PY{n}{dat\PYZus{}death} \PY{o}{=} \PY{n}{data\PYZus{}process}\PY{o}{.}\PY{n}{read\PYZus{}csv}\PY{p}{(}\PY{n}{file\PYZus{}path}\PY{p}{[}\PYZbs{}
                    \PY{l+s+s1}{\PYZsq{}}\PY{l+s+s1}{TIME\PYZus{}SERIES\PYZus{}COVID19\PYZus{}DEATHS\PYZus{}GLOBAL\PYZus{}CSV\PYZus{}PATH}\PY{l+s+s1}{\PYZsq{}}\PY{p}{]}\PY{p}{)}
\end{Verbatim}
\end{tcolorbox}

    \begin{Verbatim}[commandchars=\\\{\}]
time\_series\_covid19\_deaths\_global.csv 's information:
<class 'pandas.core.frame.DataFrame'>
RangeIndex: 266 entries, 0 to 265
Columns: 173 entries, Province/State to 7/8/20
dtypes: float64(2), int64(169), object(2)
memory usage: 359.6+ KB
None

    \end{Verbatim}

    \hypertarget{working-data}{%
\section{Working Data}\label{working-data}}

    With the constrain of time and simplicity of our model used to analys,
we decide to use only the data in EU countries that shared many
similarites in the outbreak and how goverment deal with it.

    \begin{tcolorbox}[breakable, size=fbox, boxrule=1pt, pad at break*=1mm,colback=cellbackground, colframe=cellborder]
\prompt{In}{incolor}{64}{\boxspacing}
\begin{Verbatim}[commandchars=\\\{\}]
\PY{n}{EU\PYZus{}nation} \PY{o}{=} \PYZbs{}
\PY{l+s+sd}{\PYZdq{}\PYZdq{}\PYZdq{}\PYZbs{}}
\PY{l+s+sd}{Austria, Belgium, Bulgaria, Croatia, Cyprus, \PYZbs{}}
\PY{l+s+sd}{Czechia, Denmark, Estonia, Finland, France, Germany,\PYZbs{}}
\PY{l+s+sd}{Greece, Hungary, Ireland, Italy, Latvia, Lithuania, Luxembourg,\PYZbs{}}
\PY{l+s+sd}{Malta, Netherlands, Poland, Portugal, Romania, \PYZbs{}}
\PY{l+s+sd}{Slovakia, Slovenia, Spain, Sweden \PYZbs{}}
\PY{l+s+sd}{\PYZdq{}\PYZdq{}\PYZdq{}}\PY{o}{.}\PY{n}{replace}\PY{p}{(}\PY{l+s+s2}{\PYZdq{}}\PY{l+s+s2}{ }\PY{l+s+s2}{\PYZdq{}}\PY{p}{,}\PY{l+s+s2}{\PYZdq{}}\PY{l+s+s2}{\PYZdq{}}\PY{p}{)}\PY{o}{.}\PY{n}{split}\PY{p}{(}\PY{l+s+s2}{\PYZdq{}}\PY{l+s+s2}{,}\PY{l+s+s2}{\PYZdq{}}\PY{p}{)}

\PY{c+c1}{\PYZsh{} United Kingdom did not official exit from the EU}
\PY{c+c1}{\PYZsh{} so we still consider UK is a part of EU}

\PY{n}{EU\PYZus{}nation}\PY{o}{.}\PY{n}{append}\PY{p}{(}\PY{l+s+s1}{\PYZsq{}}\PY{l+s+s1}{United Kingdom}\PY{l+s+s1}{\PYZsq{}}\PY{p}{)}
\end{Verbatim}
\end{tcolorbox}

    Dependent states is not consider in our analysis due to the geographical
distances.

    \begin{tcolorbox}[breakable, size=fbox, boxrule=1pt, pad at break*=1mm,colback=cellbackground, colframe=cellborder]
\prompt{In}{incolor}{86}{\boxspacing}
\begin{Verbatim}[commandchars=\\\{\}]
\PY{k}{for} \PY{n}{name\PYZus{}df}\PY{p}{,} \PY{n}{df} \PY{o+ow}{in} \PY{p}{[}\PY{p}{[}\PY{l+s+s1}{\PYZsq{}}\PY{l+s+s1}{EU\PYZus{}confirmed}\PY{l+s+s1}{\PYZsq{}}\PY{p}{,} \PY{n}{dat\PYZus{}confirmed}\PY{p}{]}\PY{p}{,} 
                    \PY{p}{[}\PY{l+s+s1}{\PYZsq{}}\PY{l+s+s1}{EU\PYZus{}death}\PY{l+s+s1}{\PYZsq{}} \PY{p}{,}\PY{n}{dat\PYZus{}death}\PY{p}{]}\PY{p}{,}
                    \PY{p}{[}\PY{l+s+s1}{\PYZsq{}}\PY{l+s+s1}{EU\PYZus{}recovered}\PY{l+s+s1}{\PYZsq{}}\PY{p}{,}\PY{n}{dat\PYZus{}recovered}\PY{p}{]}\PY{p}{]}\PY{p}{:}

    \PY{n}{df\PYZus{}eu} \PY{o}{=} \PY{n}{df}\PY{p}{[}\PY{n}{df}\PY{p}{[}\PY{l+s+s1}{\PYZsq{}}\PY{l+s+s1}{Country/Region}\PY{l+s+s1}{\PYZsq{}}\PY{p}{]}\PY{o}{.}\PYZbs{}
                    \PY{n}{isin}\PY{p}{(}\PY{n}{EU\PYZus{}nation}\PY{p}{)}\PY{p}{]}\PY{o}{.}\PY{n}{reset\PYZus{}index}\PY{p}{(}\PY{n}{drop}\PY{o}{=}\PY{k+kc}{True}\PY{p}{)}

    \PY{c+c1}{\PYZsh{} remove dependent state}
    \PY{n}{df\PYZus{}eu} \PY{o}{=} \PY{n}{df\PYZus{}eu}\PY{p}{[}\PY{n}{df\PYZus{}eu}\PY{p}{[}\PY{l+s+s1}{\PYZsq{}}\PY{l+s+s1}{Province/State}\PY{l+s+s1}{\PYZsq{}}\PY{p}{]}\PY{o}{.}\PY{n}{isnull}\PY{p}{(}\PY{p}{)}\PY{p}{]}\PY{o}{.}\PYZbs{}
                    \PY{n}{reset\PYZus{}index}\PY{p}{(}\PY{n}{drop}\PY{o}{=}\PY{k+kc}{True}\PY{p}{)}
    
    \PY{c+c1}{\PYZsh{} drop unnecessary columns}
    \PY{n}{df\PYZus{}eu} \PY{o}{=} \PY{n}{df\PYZus{}eu}\PY{o}{.}\PY{n}{drop}\PY{p}{(}\PY{p}{[}\PY{l+s+s1}{\PYZsq{}}\PY{l+s+s1}{Province/State}\PY{l+s+s1}{\PYZsq{}}\PY{p}{,}\PY{l+s+s1}{\PYZsq{}}\PY{l+s+s1}{Lat}\PY{l+s+s1}{\PYZsq{}}\PY{p}{,}\PY{l+s+s1}{\PYZsq{}}\PY{l+s+s1}{Long}\PY{l+s+s1}{\PYZsq{}}\PY{p}{]}\PY{p}{,}\PY{n}{axis} \PY{o}{=}\PY{l+m+mi}{1}\PY{p}{)}
                    
    \PY{n}{file}\PY{o}{.}\PY{n}{save\PYZus{}pickle}\PY{p}{(}\PY{n}{dir\PYZus{}path}\PY{p}{[}\PY{l+s+s1}{\PYZsq{}}\PY{l+s+s1}{WORKING\PYZus{}DIR}\PY{l+s+s1}{\PYZsq{}}\PY{p}{]} \PY{o}{+}\PY{l+s+s1}{\PYZsq{}}\PY{l+s+s1}{/}\PY{l+s+s1}{\PYZsq{}}\PY{o}{+} \PY{n}{name\PYZus{}df}\PY{o}{+}\PY{l+s+s1}{\PYZsq{}}\PY{l+s+s1}{.pkl}\PY{l+s+s1}{\PYZsq{}}\PY{p}{,} \PY{n}{df\PYZus{}eu}\PY{p}{)}
    \PY{n}{df\PYZus{}eu}\PY{o}{.}\PY{n}{to\PYZus{}csv}\PY{p}{(}\PY{n}{dir\PYZus{}path}\PY{p}{[}\PY{l+s+s1}{\PYZsq{}}\PY{l+s+s1}{WORKING\PYZus{}DIR}\PY{l+s+s1}{\PYZsq{}}\PY{p}{]} \PY{o}{+}\PY{l+s+s1}{\PYZsq{}}\PY{l+s+s1}{/}\PY{l+s+s1}{\PYZsq{}}\PY{o}{+} \PY{n}{name\PYZus{}df}\PY{o}{+}\PY{l+s+s1}{\PYZsq{}}\PY{l+s+s1}{.csv}\PY{l+s+s1}{\PYZsq{}}\PY{p}{,}\PY{n}{index}\PY{o}{=}\PY{k+kc}{False}\PY{p}{)}
\end{Verbatim}
\end{tcolorbox}

    \begin{Verbatim}[commandchars=\\\{\}]
Directory  /home/phuc/Desktop/Work/Data Sience/Epidemic
Modeling/MathModel/data/working  already exists
Saved file EU\_confirmed.pkl
Directory  /home/phuc/Desktop/Work/Data Sience/Epidemic
Modeling/MathModel/data/working  already exists
Saved file EU\_death.pkl
Directory  /home/phuc/Desktop/Work/Data Sience/Epidemic
Modeling/MathModel/data/working  already exists
Saved file EU\_recovered.pkl
    \end{Verbatim}

    \begin{tcolorbox}[breakable, size=fbox, boxrule=1pt, pad at break*=1mm,colback=cellbackground, colframe=cellborder]
\prompt{In}{incolor}{ }{\boxspacing}
\begin{Verbatim}[commandchars=\\\{\}]

\end{Verbatim}
\end{tcolorbox}

    \hypertarget{colab-settings}{%
\section{Colab Settings}\label{colab-settings}}

    \hypertarget{connect-drive}{%
\subsection{Connect Drive}\label{connect-drive}}

    \begin{tcolorbox}[breakable, size=fbox, boxrule=1pt, pad at break*=1mm,colback=cellbackground, colframe=cellborder]
\prompt{In}{incolor}{ }{\boxspacing}
\begin{Verbatim}[commandchars=\\\{\}]
\PY{k+kn}{from} \PY{n+nn}{google}\PY{n+nn}{.}\PY{n+nn}{colab} \PY{k+kn}{import} \PY{n}{drive} 
\PY{c+c1}{\PYZsh{} conncent phuc.phan1012000@hcmut.edu.vn}

\PY{c+c1}{\PYZsh{} default location for the drive}
\PY{n}{ROOT} \PY{o}{=} \PY{l+s+s2}{\PYZdq{}}\PY{l+s+s2}{/content/drive}\PY{l+s+s2}{\PYZdq{}}            
\PY{n}{drive}\PY{o}{.}\PY{n}{mount}\PY{p}{(}\PY{n}{ROOT}\PY{p}{)}   
\end{Verbatim}
\end{tcolorbox}

    \hypertarget{load-library}{%
\subsection{Load Library}\label{load-library}}

    \begin{tcolorbox}[breakable, size=fbox, boxrule=1pt, pad at break*=1mm,colback=cellbackground, colframe=cellborder]
\prompt{In}{incolor}{ }{\boxspacing}
\begin{Verbatim}[commandchars=\\\{\}]
\PY{k+kn}{import} \PY{n+nn}{sys}
\PY{k+kn}{import} \PY{n+nn}{os}
\PY{c+c1}{\PYZsh{} Load phuc\PYZsq{}s library on Colab}
\PY{n}{sys}\PY{o}{.}\PY{n}{path}\PY{o}{.}\PY{n}{append}\PY{p}{(}\PY{l+s+s2}{\PYZdq{}}\PY{l+s+s2}{/content/drive/My Drive/Work/Data\PYZus{}Science\PYZus{}Library}\PY{l+s+s2}{\PYZdq{}}\PY{p}{)}
\end{Verbatim}
\end{tcolorbox}

    \hypertarget{setup-project}{%
\subsection{Setup Project}\label{setup-project}}

    \begin{tcolorbox}[breakable, size=fbox, boxrule=1pt, pad at break*=1mm,colback=cellbackground, colframe=cellborder]
\prompt{In}{incolor}{ }{\boxspacing}
\begin{Verbatim}[commandchars=\\\{\}]
\PY{n}{GIT\PYZus{}REPOSITORY} \PY{o}{=} \PY{l+s+s2}{\PYZdq{}}\PY{l+s+s2}{kalapa\PYZhy{}s\PYZhy{}CreditScore}\PY{l+s+s2}{\PYZdq{}}
\PY{c+c1}{\PYZsh{} path to project on Google Drive}
\PY{n}{MY\PYZus{}GOOGLE\PYZus{}DRIVE\PYZus{}PATH} \PY{o}{=} \PY{l+s+s1}{\PYZsq{}}\PY{l+s+s1}{My Drive/Work/Projects}\PY{l+s+s1}{\PYZsq{}}
\PY{n}{PROJECT\PYZus{}PATH} \PY{o}{=} \PY{n}{os}\PY{o}{.}\PY{n}{path}\PY{o}{.}\PY{n}{join}\PY{p}{(}\PY{n}{ROOT}\PY{p}{,} \PY{n}{MY\PYZus{}GOOGLE\PYZus{}DRIVE\PYZus{}PATH}\PY{p}{,}\PY{n}{GIT\PYZus{}REPOSITORY}\PY{p}{)}

\PY{c+c1}{\PYZsh{} GO TO WORKING PROJECT PATH}
\PY{n}{os}\PY{o}{.}\PY{n}{chdir}\PY{p}{(}\PY{n}{PROJECT\PYZus{}PATH}\PY{o}{+}\PY{l+s+s1}{\PYZsq{}}\PY{l+s+s1}{/src}\PY{l+s+s1}{\PYZsq{}}\PY{p}{)}
\PY{n}{os}\PY{o}{.}\PY{n}{remove}\PY{p}{(}\PY{n}{PROJECT\PYZus{}PATH}\PY{o}{+}\PY{l+s+s1}{\PYZsq{}}\PY{l+s+s1}{/data\PYZus{}path.pkl}\PY{l+s+s1}{\PYZsq{}}\PY{p}{)}
\end{Verbatim}
\end{tcolorbox}


    % Add a bibliography block to the postdoc
    
    
    
\end{document}
